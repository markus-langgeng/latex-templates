\begin{luacode*}
require"chn-to-pinyin"

pinyin_vowel_v_mapping = {
    "ü", "Ü"
}

pinyin_vowel_tone_mapping = {
    ["a"] = {
        "ā", "á", "ǎ", "à"
    },
    ["e"] = {
        "ē", "é", "ě", "è"
    },
    ["i"] = {
        "ī", "í", "ǐ", "ì"
    },
    ["o"] = {
        "ō", "ó", "ǒ", "ò"
    },
    ["u"] = {
        "ū", "ú", "ǔ", "ù"
    },
    ["v"] = {
        "ǖ", "ǘ", "ǚ", "ǜ"
    },
    ["A"] = {
        "Ā", "Á", "Ǎ", "À"
    },
    ["E"] = {
        "Ē", "É", "Ě", "È"
    },
    ["I"] = {
        "Ī", "Í", "Ǐ", "Ì"
    },
    ["O"] = {
        "Ō", "Ó", "Ǒ", "Ò"
    },
    ["U"] = {
        "Ū", "Ú", "Ǔ", "Ù"
    },
    ["V"] = {
        "Ǖ", "Ǘ", "Ǚ", "Ǜ"
    }
}

pinyin_vowel_letters_lower = {"a", "e", "i", "o", "u", "v"}

pinyin_vowel_to_lower = {}
pinyin_vowel_to_upper = {}

for ind, lower in ipairs(pinyin_vowel_letters_lower) do
    local upper = lower:upper()
    local tbl1 = pinyin_vowel_tone_mapping[lower]
    local tbl2 = pinyin_vowel_tone_mapping[upper]
    assert(#tbl1 == #tbl2, "inconsistent pinyin configuration")
    for i=1,#tbl1 do
        pinyin_vowel_to_lower[tbl2[i]] = tbl1[i]
        pinyin_vowel_to_upper[tbl1[i]] = tbl2[i]
    end
end
pinyin_vowel_to_upper[pinyin_vowel_v_mapping[1]] = pinyin_vowel_v_mapping[2]
pinyin_vowel_to_lower[pinyin_vowel_v_mapping[2]] = pinyin_vowel_v_mapping[1]

pinyin_consonants = {
    "zh", "ch", "sh", "b", "p", "m", "f", "d",
    "t", "n", "l", "g", "k", "h", "j", "q", "x",
    "r", "z", "c", "s", "w", "y"
}

pinyin_vowels = {
    "iang", "uang", "ueng", "iong", "uai", "uei", "iao", "ian",
    "uan", "van", "ang", "eng", "ing", "ong", "ia", "ua", "uo",
    "ie", "ve", "ai", "ei", "ao", "ou", "iu", "an", "en", "in",
    "un", "vn", "ue", "er", "i", "u", "v", "a", "o", "e"
}

pinyin_vowel_tone_locs = {
    2, 2, 2, 2, 2, 2, 2, 2, 2, 2, 1, 1, 1, 1, 2, 2, 2, 2, 2,
    1, 1, 1, 1, 1, 1, 1, 1, 1, 1, 2, 1, 1, 1, 1, 1, 1, 1
}

assert(#pinyin_vowels == #pinyin_vowel_tone_locs, "inconsistent pinyin configuration")

function find_next_consonant(s, s_lower, loc)
    local b
    for _, consonant in ipairs(pinyin_consonants) do
        b, _ = s_lower:find(consonant, loc)
        if b == loc then
            return s:sub(loc, loc + #consonant - 1)
        end
    end
    return nil
end

function find_next_vowel(s, s_lower, loc)
    local b
    for vowel_ind, vowel in ipairs(pinyin_vowels) do
        b, _ = s_lower:find(vowel, loc)
        if b == loc then
            return s:sub(loc, loc + #vowel - 1), vowel_ind
        end
    end
    return nil
end

function find_next_tone(s, s_lower, loc)
    local char = s_lower:sub(loc, loc)
    if char ~= nil and char:len() > 0 then
        local byte = char:byte()
        if byte >= 48 and byte <= 57 then
            return char
        end
    end
    return nil
end

function render_pinyin(consonant, vowel, vowel_ind, tone)
    local res = ""
    if consonant ~= nil then
        res = res .. consonant
    end
    local tone_loc = -1
    if tone ~= nil then
         tone_loc = pinyin_vowel_tone_locs[vowel_ind]
    end
    local char, tone_char
    for i=1,vowel:len() do
        char = vowel:sub(i, i)
        if i == tone_loc then
            tone_char = pinyin_vowel_tone_mapping[char][tonumber(tone)]
            assert(tone_char ~= nil, string.format("invalid tone combination: '%s', %s", vowel, tone))
            res = res .. tone_char
        else
            -- deal with "v"
            if char == "v" then
                res = res .. pinyin_vowel_v_mapping[1]
            elseif char == "V" then
                res = res .. pinyin_vowel_v_mapping[2]
            else
                res = res .. char
            end
        end
    end
    return res
end

function latin_to_pinyin(s)
    -- discard spaces
    s = s:gsub("%s", "")
    local s_lower = s:lower()
    local loc = 1
    local old_loc = loc
    local consonant, vowel, tone, vowel_ind, render
    local result = {}
    while loc <= s:len() do
        -- start by finding consonant
        consonant = find_next_consonant(s, s_lower, loc)
        if consonant ~= nil then
            loc = loc + consonant:len()
        end
        -- proceed to find vowel
        vowel, vowel_ind = find_next_vowel(s, s_lower, loc)
        assert(vowel ~= nil, string.format("cannot find vowel for '%s...'", s:sub(loc, loc + 10)))
        loc = loc + vowel:len()
        -- peek next char to see if we can find a tone
        tone = find_next_tone(s, s_lower, loc)
        if tone ~= nil then
            loc = loc + tone:len()
        end
        if loc == old_loc then
            error("latin to pinyin algorithm is stuck, please check the integrity of latin input")
        end
        texio.write_nl(string.format("pinyin-debug: %s, %s, %s", consonant, vowel, tone))
        render = render_pinyin(consonant, vowel, vowel_ind, tone)
        table.insert(result, render)
        old_loc = loc
    end
    return result
end

function chinese_to_pinyin(s)
    local code = utf8.codepoint(s)
    local char = utf8.char(code)
    local pinyin = chn_to_pinyin[char]
    return char, pinyin
end


function pinyin_upper(s)
    cap_first = cap_first or false
    local res = ""
    local query, char
    for p, code in utf8.codes(s) do
        char = utf8.char(code)
        query = pinyin_vowel_to_upper[char]
        if query ~= nil then
            res = res .. query
        else
            res = res .. char:upper()
        end
    end
    return res
end

function chinese_to_pinyin_latex(s, cap_first)
    cap_first = cap_first or false
    local char, pinyin = chinese_to_pinyin(s)
    local out_val  = ""
    if pinyin==nil then
        -- special treatment for "*"
        if char == "*" then
            out_val = char
        else
            out_val = [[\textcolor{red}{]] .. char .. "}"
        end
    else
        if cap_first then
            local first_code = utf8.codepoint(pinyin)
            local first_code_offset = utf8.offset(pinyin, 2)
            pinyin = pinyin_upper(utf8.char(first_code)) .. pinyin:sub(first_code_offset)
        end
        out_val = pinyin
    end
    token.set_macro("l_doc_tmpc_tl", out_val)
end



\end{luacode*}

\newcommand{\latintopinyin}[1]{%
    \directlua{
        local pinyin_table = latin_to_pinyin("\luaescapestring{#1}")
        tex.print(table.concat(pinyin_table, " "))
    }%
}

\ExplSyntaxOn

\tl_new:N \l_doc_tmpa_tl
\tl_new:N \l_doc_tmpb_tl
\tl_new:N \l_doc_tmpc_tl

\NewDocumentCommand{\chntopinyin}{sm}{
    \group_begin:
    \tl_set:Nn \l_doc_tmpa_tl {#2}
    \tl_clear:N \l_doc_tmpb_tl
    \bool_do_until:nn {\tl_if_empty_p:N \l_doc_tmpa_tl} {
        \exp_args:NV \tl_if_head_is_group:nTF \l_doc_tmpa_tl {
            \tl_put_right:Nx \l_doc_tmpb_tl {{\tl_head:N \l_doc_tmpa_tl}}
        } {
            \exp_args:Nx \token_if_cs:NTF {\tl_head:N \l_doc_tmpa_tl} {
                \tl_put_right:Nx \l_doc_tmpb_tl {\tl_head:N \l_doc_tmpa_tl}
            } {
                \IfBooleanTF{#1}{
                    \directlua{
                        chinese_to_pinyin_latex("\luaescapestring{\tl_head:N\l_doc_tmpa_tl}", true)
                    }
                    \tl_put_right:NV \l_doc_tmpb_tl \l_doc_tmpc_tl
                }{
                    \directlua{
                        chinese_to_pinyin_latex("\luaescapestring{\tl_head:N\l_doc_tmpa_tl}")
                    }
                    \tl_put_right:NV \l_doc_tmpb_tl \l_doc_tmpc_tl
                }
                \tl_put_right:NV \l_doc_tmpb_tl \space
            }
        }
        \tl_set:Nx \l_doc_tmpa_tl {\tl_tail:N \l_doc_tmpa_tl}
    }
    \tl_use:N \l_doc_tmpb_tl
    \group_end:
}


\newcommand{\pinyin}[2]{
    \latintopinyin{#2}\space
}

\ExplSyntaxOff
